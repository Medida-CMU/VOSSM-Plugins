\nonstopmode{}
\documentclass[letterpaper]{book}
\usepackage[times,inconsolata,hyper]{Rd}
\usepackage{makeidx}
\usepackage[utf8,latin1]{inputenc}
% \usepackage{graphicx} % @USE GRAPHICX@
\makeindex{}
\begin{document}
\chapter*{}
\begin{center}
{\textbf{\huge Package `scimapClient'}}
\par\bigskip{\large \today}
\end{center}
\begin{description}
\raggedright{}
\item[Type]\AsIs{Package}
\item[Title]\AsIs{Collects and Reports Usage Data on R Packages and store it in a
file}
\item[Version]\AsIs{0.2.1}
\item[Date]\AsIs{2015-011-01}
\item[Author]\AsIs{Christopher Bogart}
\item[Maintainer]\AsIs{Christopher Bogart }\email{cbogartdenver@gmail.com}\AsIs{}
\item[Extension]\AsIs{Anandh Somasundaram <yesyayen@gmail.com>}
\item[Description]\AsIs{This package sends anonymized package usage information to
http://scisoft-net-map.isri.cmu.edu. You can help the creators of scientific
software packages show the usage and impact of their software to employers,
funding agencies, tenure committees, etc.}
\item[URL]\AsIs{}\url{http://scisoft-net-map.isri.cmu.edu}\AsIs{}
\item[License]\AsIs{BSD\_2\_clause + file LICENSE}
\item[Imports]\AsIs{RJSONIO, tools, utils}
\item[Suggests]\AsIs{testthat}
\end{description}
\Rdcontents{\R{} topics documented:}
\inputencoding{utf8}
\HeaderA{addUserMetadata}{Add user metadata}{addUserMetadata}
%
\begin{Description}\relax
Upload custom metadata to \code{scisoft-net-map.isri.cmu.edu}
\end{Description}
%
\begin{Usage}
\begin{verbatim}
addUserMetadata(metadata)
\end{verbatim}
\end{Usage}
%
\begin{Arguments}
\begin{ldescription}
\item[\code{metadata}] An R object (a list with named elements);
this will be converted to JSON by \code{\LinkA{toJSON}{toJSON}}
\end{ldescription}
\end{Arguments}
%
\begin{Details}\relax
The \code{scimapClient} package automatically uploads R usage and dependency
information about the packages you used, at the end of each R session,
or at the next prompt after an hour of idle time.

These two functions optionally attach metadata to that packet.

The \code{addUserMetadata} function lets you supply arbitrary information to be
associated with each R session.  It could be used to implement an online lab notebook,
helping you navigate and interpret an online record of your own work in R.
If you wish to keep your data anonymous, don't put anything in the metadata
that will allow others to identify you.
\end{Details}
%
\begin{Examples}
\begin{ExampleCode}
## Not run: addUserMetadata(list(project="Arrow trajectory analysis", version="v5.4"))
\end{ExampleCode}
\end{Examples}
\inputencoding{utf8}
\HeaderA{deanonymize}{Associate personal identity with usage data}{deanonymize}
%
\begin{Description}\relax
Associate personal identity with usage data
\end{Description}
%
\begin{Usage}
\begin{verbatim}
deanonymize(name, webpage, pubspage)
\end{verbatim}
\end{Usage}
%
\begin{Arguments}
\begin{ldescription}
\item[\code{name}] Your name, as shown in author lists (if you publish papers),
or "" if you don't want to give a name

\item[\code{webpage}] A website about you,
or "" if you don't want to give a site

\item[\code{pubspage}] A website that lists your publications,
such as an academic publications list page
or your Google Scholar citations page;
or "" if you don't want to give a site
\end{ldescription}
\end{Arguments}
%
\begin{Details}\relax
Adds name, webpage, and publications page metadata to the usage packet.
You could do this with \code{\LinkA{addUserMetadata}{addUserMetadata}}; this function simply
uses standard metadata field names to do so.

We (the operators of \code{scisoft-net-map.isri.cmu.edu}) may infer the same user for other
R sessions run by the same R installation.  Deanonymizing is optional,
and you can fill just some of the fields if you like.

The advantage of deanonymizing is that you can help package authors find
literature that may have relied on their package. This can help
them justify the time they spend maintaining packages, or help them
allocate their time to supporting packages that are most useful to scientists.
\end{Details}
%
\begin{References}\relax
See your usage and others' at \url{http://scisoft-net-map.isri.cmu.edu}
\end{References}
%
\begin{SeeAlso}\relax
\code{\LinkA{previewPacket}{previewPacket}}
\end{SeeAlso}
%
\begin{Examples}
\begin{ExampleCode}
library(scimapClient)
## Not run: deanonymize("Chris Bogart", "http://quetzal.bogarthome.net/",
    "http://scholar.google.com/citations?user=FQSWa4sAAAAJ&hl=en")
## End(Not run)
\end{ExampleCode}
\end{Examples}
\inputencoding{utf8}
\HeaderA{enableScimap}{Enable or revoke permission for usage tracking}{enableScimap}
\aliasA{disableScimap}{enableScimap}{disableScimap}
\aliasA{disableScimapThisSession}{enableScimap}{disableScimapThisSession}
%
\begin{Description}\relax
Enable or revoke permission for usage tracking
\end{Description}
%
\begin{Usage}
\begin{verbatim}
enableScimap()

disableScimap()

disableScimapThisSession()
\end{verbatim}
\end{Usage}
%
\begin{Details}\relax
Enable or revoke permission for the package to
send usage statistics to \code{scisoft-net-map.isri.cmu.edu}.

This is run by hand in interactive mode; there is no
need to put in a script.  Once you have revoked permission with \code{disableScimap},
the scimapClient package will not track your usage again
until you run \code{enableScimap()}

You can temporarily revoke permission (just until R is stopped and restarted)
with \code{disableScimapThisSession}.

The \code{enableScimap} function saves code in your .Rprofile that
automatically loads this package each time you run R, and sets a random unique
ID that anonymizes your data.

\code{disableScimap} removes that code from your .Rprofile.
\end{Details}
%
\begin{Section}{Functions}
\begin{itemize}

\item \code{}: Permanently disable sending of packets

\item \code{}: Disable sending packets just for this session (until R is closed and reopened)

\end{itemize}
\end{Section}
%
\begin{References}\relax
See your usage and others' at \url{http://scisoft-net-map.isri.cmu.edu}
\end{References}
%
\begin{SeeAlso}\relax
\code{\LinkA{previewPacket}{previewPacket}}
\end{SeeAlso}
\inputencoding{utf8}
\HeaderA{enableTracking}{Enable tracking for this session}{enableTracking}
%
\begin{Description}\relax
Set the installation ID and enable tracking for this session
\end{Description}
%
\begin{Usage}
\begin{verbatim}
enableTracking(randomID)
\end{verbatim}
\end{Usage}
%
\begin{Arguments}
\begin{ldescription}
\item[\code{randomID}] Should be a random string that is stable for a particular
R user.
\end{ldescription}
\end{Arguments}
%
\begin{Details}\relax
This should be called from the user's .Rprofile; ideally the
randomID value passed in should always be the same for a particular R
user.

Normally you should not need to write this function call; use enableScimap()
to generate the code to insert in .Rprofile.
\end{Details}
\inputencoding{utf8}
\HeaderA{getScimapId}{Get scimap unique/anonymous ID for your installation of R}{getScimapId}
%
\begin{Description}\relax
Return your unique installation ID for scimap usage tracking
\end{Description}
%
\begin{Usage}
\begin{verbatim}
getScimapId()
\end{verbatim}
\end{Usage}
%
\begin{Details}\relax
The scimapClient package identifies each installation of R with
a unique ID to track usage/installation statistics.
\end{Details}
%
\begin{Value}
Returns a string of 25 decimal digits.  This
is a random but fixed number with no meaning.
except as a unique identifier.
\end{Value}
%
\begin{References}\relax
See your usage and others' at \url{http://scisoft-net-map.isri.cmu.edu}
\end{References}
\inputencoding{utf8}
\HeaderA{isEnabledScimap}{Does \code{scimapClient} have permission to track usage?}{isEnabledScimap}
%
\begin{Description}\relax
Checks if the package has permission to track usage.
\end{Description}
%
\begin{Usage}
\begin{verbatim}
isEnabledScimap()
\end{verbatim}
\end{Usage}
%
\begin{Details}\relax
Permission can be granted or revoked with \code{\LinkA{enableScimap}{enableScimap}()}
or \code{disableScimap()} from the interactive prompt.

If permission is not given, the \code{scimapRegister()} function
returns silently without doing anything.  If permission is
granted, the package sends usage information, but also
returns silently.

Tracking is automatically disabled if the environment variable R\_TESTS is
set (even to ""); this prevents spurious packets from being sent during testing.
It's also disabled while packages are being installed (by checking for the
environment variable R\_PACKAGE\_NAME).
\end{Details}
%
\begin{Value}
boolean
\end{Value}
%
\begin{References}\relax
See your usage and others' at \url{http://scisoft-net-map.isri.cmu.edu}
\end{References}
%
\begin{SeeAlso}\relax
\code{\LinkA{previewPacket}{previewPacket}}
\end{SeeAlso}
%
\begin{Examples}
\begin{ExampleCode}
## Type this in interactive mode to see whether
## the scimapClient package is enabled to collect usage data.
isEnabledScimap()
# returns TRUE or FALSE
\end{ExampleCode}
\end{Examples}
\inputencoding{utf8}
\HeaderA{previewPacket}{Assemble usage information for transmission to \url{http://scisoft-net-map.isri.cmu.edu}}{previewPacket}
%
\begin{Description}\relax
Packages up anonymous usage tracking information about R packages
to a server which allows authors of packages to track how
widely used and installed they are.  This is helpful
for demonstrating the usefulness of these packages to
their employers and funding agencies.
\end{Description}
%
\begin{Usage}
\begin{verbatim}
previewPacket()
\end{verbatim}
\end{Usage}
%
\begin{Details}\relax
You can call this function after running some of your
own code in order to see what kind of information scimapClient
will send to scisoft-net-map.isri.cmu.edu (as a packet over
a TCP/IP connection, to port 7778). If you don't like what
you see, you can disable the send by typing disableScimap() to
turn off this monitoring library.

This tracking is voluntary and anonymous.  To enable tracking
type \code{enableScimap()} from the interactive prompt; to disable
it type \code{disableScimap()}.  If tracking is disabled, this
function will do nothing.

Creates a JSON record.

The record contains the following information:
\begin{description}

\item[account:]  A string of 25 random digits unique to this installation of R 
\item[job:]  The account number plus a start time 
\item[startEpoch:]  The current time 
\item[platform:]  The operating system, version, and hardware type 
\item[packages:]  The output of system(), and packageVersion() for each package listed. 
\item[dependencies:]  Static and dynamic dependencies between functions and packages 

\end{description}

\end{Details}
%
\begin{Value}
Returns an object representing the information that would be sent
\end{Value}
%
\begin{References}\relax
See your usage and others' at \url{http://scisoft-net-map.isri.cmu.edu}
\end{References}
%
\begin{SeeAlso}\relax
\code{\LinkA{enableScimap}{enableScimap}}
\end{SeeAlso}
\inputencoding{utf8}
\HeaderA{scimapClient}{Collects and reports usage data on R packages}{scimapClient}
\aliasA{scimapClient-package}{scimapClient}{scimapClient.Rdash.package}
%
\begin{Description}\relax
Collects and reports usage data on R packages
\end{Description}
%
\begin{Details}\relax
Authors of scientific software packages do not always have
good ways of documenting how widely used their work is, and
understanding how it is used in conjunction with other packages.

\code{scimapClient} allows users of these packages to help:
it sends anonymous usage tracking information about the R packages
you use to a server which allows anyone to see how
widely used and installed these packages are.  This is helpful
for demonstrating the usefulness of these packages to
their employers and funding agencies.

This tracking is voluntary and anonymous.  To enable tracking
for all future sessions, type \code{\LinkA{enableScimap}{enableScimap}()} from the
interactive prompt; to disable tracking in this and future
sessions, type \code{\LinkA{disableScimap}{disableScimap}()}.  If tracking is disabled, this
package will do nothing.
\end{Details}
%
\begin{References}\relax
See your usage and others' at \url{http://scisoft-net-map.isri.cmu.edu}
\end{References}
%
\begin{Examples}
\begin{ExampleCode}
##  Call this to reenable scimap
## Not run: enableScimap()

##  Call this to toggle scimap off if you do not want to send data.
## Not run: disableScimap()

## Call this to see what kind of information is sent
previewPacket()
\end{ExampleCode}
\end{Examples}
\printindex{}
\end{document}
